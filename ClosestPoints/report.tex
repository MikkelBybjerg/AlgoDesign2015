\documentclass{article}
\usepackage{amsmath}
\usepackage[utf8]{inputenc}

\pagestyle{empty}


\title{Closest Pair Report}
\author{Alina-Florentina Dumitrache, \\George Valentin Iuriet \\and Mikkel Bybjerg Christophersen}

\begin{document}
  \maketitle

  \section{Results}

  Our implementation produces the expected results on all input-output file pairs tested.\\
  We measure equality of solutions by whether the floating-point difference between the wanted and acquired solutions is less than $\frac{1}{10^6}$, which we consider precise enough for this assignment. \\

  The following table shows the closest pairs in the input files {\tt wc-instance-*.txt}.
  Here $n$ denotes the number of points in the input,
  and $(u,v)$ denotes a closest pair of points at distance $\delta$.

  \bigskip\noindent
  \begin{tabular}{rrrr}
    $n$ & $u$ & $v$ & $\delta$ \\
    2 & 0 & 1 & 1 \\
    6 & 0 & 4 & 1 \\
    14 & 9 & 11 & 1 \\
    30 & 9 & 11 & 1 \\
    62 & 11 & 24 & 1 \\
    126 & 91 & 122 & 1 \\
    254 & 29 & 16 & 1 \\
    1022 & 337 & 1005 & 1 \\
    4094 & 3953 & 2802 & 1 \\
    16382 & 11670 & 1043 & 1 \\
    65534 & 35326 & 14974 & 1 \\
  \end{tabular}


  \section{Implementation details}

  We resort by $y$-coordinates in each recursive step.

  For the comparison of points close to $s$ in $S_y$ we inspect 11 points,
  as explained by Troels Bjerre Sørensen in lecture, and accompanying slides by Kevin Wayne.\\	
  Here is the corresponding part of our code:
  \begin{verbatim}
    minIndex = index of minimum solution for the recursive calls
    minDist = value of minimum solution for the recursive calls
    band = list of points less than distance delta from splitting line
    
    for (int i=0;i<band.length;i++)
        for(int j=1;j<11;j++)
        {
            if(i+j >= band.length)
                continue;
            double dist = euclid(band[i], band[i+j]);
            if(dist < minDist)
            {
                minIndex = new int[]{i, i+j}; //simplified
                minDist = dist;
            }
        }
  \end{verbatim}
	
  Our solution runs in time $O(n \log^2 n)$, as $\log n$ recursive levels take place, and at each level all subarrays are sorted with respect to the y-coordinate anew, equivalent to the entire array being sorted at each level in time $O(n \log n)$. Using the merge-sort strategy of each recursion returning its y-sorted subarray, sorting could have been done in linear time, and running time could have been made $O(n\log n)$.


\end{document}
